% Template for ICIP-2012 paper; to be used with:
%          spconf.sty  - ICASSP/ICIP LaTeX style file, and
%          IEEEbib.bst - IEEE bibliography style file.
% --------------------------------------------------------------------------
\documentclass{article}
\usepackage{spconf,amsmath,graphicx}

% Example definitions.
% --------------------
\def\x{{\mathbf x}}
\def\L{{\cal L}}

% Title.
% ------
\title{OpenDcd: Open Source WFST Decoding Toolkit}
%
% Single address.
% ---------------
\name{Author(s) Name(s)}
\address{Author Affiliation(s)}
%
% For example:
% ------------
%\address{School\\
%	Department\\
%	Address}
%
% Two addresses (uncomment and modify for two-address case).
% ----------------------------------------------------------
%\twoauthors
%  {A. Author-one, B. Author-two\sthanks{Thanks to XYZ agency for funding.}}
%	{School A-B\\
%	Department A-B\\
%	Address A-B}
%  {C. Author-three, D. Author-four\sthanks{The fourth author performed the work
%	while at ...}}
%	{School C-D\\
%	Department C-D\\
%	Address C-D}
%
\begin{document}
%\ninept
%
\maketitle
%
\begin{abstract}
In this paper we introduce \emph{OpenDcd} a lightweight and portable Weighted Finite State
Transducer based speech decoding toolkit. The toolkit is faster and requires
substantially less memory than other open source alternatives.
\end{abstract}
%
\begin{keywords}
WFST, Decoding, Speech recognition, Open source
\end{keywords}
%
\section{Introduction}
\label{sec:intro}


Open source and reproducible results are speech recognition are an essential
part of modern science. Recently, there has become available many excellent
tools for manipulating automata, building language models, all the way up
build sophisticated speech recognitions system.  However, the state of modern 
decoding libraries is still lagging. In this paper we describe the OpenDcd release. 
One extremely important feature of the toolkit is it can be built as 
The toolkit can be built as standalone.

\section{Introduction}
\label{sec:format}

\section{Decoder Core}
\label{sec:decodercore}

\subsection{Customization}
\label{sec:}

\subsection{Instrumentation}
\label{sec:instrumentation}

\section{Cascade Tools}
\label{sec:cascade}

\section{Post-processing Tools}
\label{sec:postprocess}

\section{Evaluations}
\label{sec:majhead}

\subsection{Wall Street Journal}

\subsection{Voice Search}

\section{Summary}
\label{sec:page}
In this paper we have described OpenDcd a toolkit.

\bibliographystyle{IEEEbib}
\bibliography{strings,refs}

\end{document}
