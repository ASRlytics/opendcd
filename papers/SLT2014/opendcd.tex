% Template for ICIP-2012 paper; to be used with:
%          spconf.sty  - ICASSP/ICIP LaTeX style file, and
%          IEEEbib.bst - IEEE bibliography style file.
% --------------------------------------------------------------------------
\documentclass{article}
\usepackage{spconf,amsmath,graphicx,minted,url}

% Example definitions.
% --------------------
\def\x{{\mathbf x}}
\def\L{{\cal L}}

% Title.
% ------
\title{OpenDcd: Open Source WFST Decoding Toolkit}
%
% Single address.
% ---------------
\name{Paul R. Dixon \quad Josef R. Novak}
\address{Yandex Zurich}
%
% For example:
% ------------
%\address{School\\
%	Department\\
%	Address}
%
% Two addresses (uncomment and modify for two-address case).
% ----------------------------------------------------------
%\twoauthors
%  {A. Author-one, B. Author-two\sthanks{Thanks to XYZ agency for funding.}}
%	{School A-B\\
%	Department A-B\\
%	Address A-B}
%  {C. Author-three, D. Author-four\sthanks{The fourth author performed the work
%	while at ...}}
%	{School C-D\\
%	Department C-D\\
%	Address C-D}
%
\begin{document}
%\ninept
%
\maketitle
%
\begin{abstract}
In this paper we introduce \emph{OpenDcd} a lightweight and portable Weighted Finite State
Transducer based speech decoding toolkit written in C++. OpenDcd is a collection or tools for speech
recognition decoding, cascade construction and related conversion and results manipulation.
The toolkit is faster and requires substantially less memory than other open source alternatives.
We hope this toolkit rescues from inefficient full $HCLG$ compilations.

Through sophisticated use of templates OpenDcd can be built against Kaldi and re-use the
frontend and acoustic scoring. This Allows OpenDcd to be used easily as a complete speech recognition system.

\end{abstract}
%
\begin{keywords}
WFST, Decoding, Speech recognition, Open source
\end{keywords}
%
\section{Introduction}
\label{sec:intro}
Open source and reproducible results are speech recognition are an essential
part of modern science. Recently, there has become available many excellent
tools for manipulating automata, building language models, all the way up
build sophisticated speech recognitions system.  However, the state of modern 
decoding libraries is still lagging. In this paper we describe the OpenDcd
release a set of tools under active development for constructing and decoding
finite state speech recognition cascades. The toolkit makes use of OpenFst for the 
underlying finite state representations
and operations, and the OpenGrm~\cite{roark12} for language model conversion.

One extremely important feature of the toolkit is it can be built as 
The toolkit can be built as standalone decoder with no dependencies, or as addon
with Kaldi. For the latter there is full access to all of the IO mechanism and 
acoustic models available in Kaldi. In later sections we discuss the lattice generation
mechanisms that make it possible to use the OpenDcd in many of Kaldi's training and
rescoring pipelines. 

This means if you have a set of converted Kaldi models no additional software needs
to be installed. OpenDcd without Kaldi can be deployed to any platform and the
recognition system run. 

Furthermore the modular nature of OpenDcd's  design allow for new types of 
acoustic model for example LSTM or RNN to plugged into the decoder. 

The library started off as an internal project for research use we are
releasing the project under an open source licence. One design goal was 
to allow for customization via \texttt{C++} templates.

The focus of the toolkit is provide a fast easy-to-use WFST decoder,
with minimal dependencies that can be used as a tool research or
as part of a larger software system.

\section{Overview}
\label{sec:format}
There are several components in the system. The cascade tools, the decoder core
and utility tools. In addition we provide a set of helpful scripts and Kaldi style
example that show how to integrate the decoder into a standard Kaldi reciee.


\subsection{Weighted Finite State Transducers}


\section{Decoder Core}
\label{sec:decodercore}

\subsection{Transition Model}
The search component assumes that the transition model\footnote{Our model and definition of TransitionModel
are different to that used in Kaldi.} will consume a single
frame of speech per time step. The transition module (nearly) full encapsulates
the acoustic model. The decoupling allows for specialized token expansion algorithm
to handle any topolgy of HMM or even allow for totally arbitrary non-finite state
models. 


The recognition output is written as either an OpenFst FAR file or a Kaldi table. Each of which can
stored in the tropical semiring or a Kaldi lattice weight. In addition tools are provided to convert
between FAR and table formats. To batch manipulate the results we provided a reimplementation of
the AT\&T \texttt{farfilter} command. \texttt{farprintnbeststrings} adds AT\&T functionality missing
from OpenFst that is the ability to print an nbest list from lattices contained within a FAR file,
we make use of two different algoritms. 


\subsection{Customization}
\label{sec:custom}
The core of the decoder is the \texttt{ArcDecoder} class that is parametrized on a transition
model, Fst  and lattice.
\begin{minted}[mathescape,
               numbersep=5pt,
               gobble=2,
               frame=lines,
               framesep=2mm]{cpp}
  template<class T, class F, class L>
  class ArcDecoder;
\end{minted}

For example the following code in Kaldi mode will create a decoder that uses
cascade in the tropical semiring, has HMM transition model with GMM state 
output distributions and generates lattice.

\begin{minted}[mathescape,
               numbersep=5pt,
               gobble=2,
               frame=lines,
               framesep=2mm]{cpp}
  typedef HMMTransitionModel<DecodableAmNnet> 
    ObservationModel;
  ArcDecoder<StdFst,         
             ObservationModel,
             SimpleLattice> Decoder;
\end{minted}

\subsection{Lattice Generation}
The decoder supports lattice generation based on the \emph{phone-pair}
approximation and a \emph{full} lattice generation strategy. In addition 
there a two approaches for generating full state level lattices. The first
is compose a HMM level.

The decoder can be simply configured to generate lattice any of the below formats
Tropical, Log, Lattice and CompactLattice. Not there are certain restrictions.
In addtion the toolkit provides the code necesssary to build dynamic shared objects
that allow the OpenFst tools to work with Kaldi's custom weight semirings.

\subsection{Parallelization}
The recent work on acoustic lookahead and parallelization only focus on dynamic decoders
or WFST decoder that use lookahead composition. We introduce a different parallelization
scheme that uses a worker thread to pre-emp upcoming states and precompute the composition
state. In practise we find that this heuristic can give an approximately 20\% speed-up.

\subsection{Factorization}
The de-coupled transition model implemenation not allows for non HMM based models to be easily
integrated into the decoder but allows for arbitrary topologies that may be the result of more
complex models or more complex factorizations of the search space.

%\url{https://code.google.com/p/shinyprofiler/}

\section{Instrumentation and Logging}
\label{sec:instrumentation}
The decoder has several built analysis mechanism, these are the call analyzer,
memory analyzer and the search space analyzer. For fined grained analysis we 
make use of the ShinyProfiler which generate very detailed call graphs.

\section{Cascade Tools}
\label{sec:cascade}
The toolkit includes a very fast build procedure for constructing a recognition
\emph{cascade} from a language, lexicon and context-dependency models. In 
machine translation experimental pipelines have become more popular~\cite{koehn10}. 
However, most of these are very similar to well known Unix make tool. We make 
use of \texttt{make} as our basic tool to track the components. 
If a compoenent in the cascade is modified make we only rebuild.

The FST framework provides many operations for combining and manipulating
automata. To construct a speech recognition cascade and recognize speech 
typically requires three core operations: Composition, Determinization and
Shortestpath. The later is handled be the decoder command and by careful
component construction. We 

The ngram model uses the direct minimalistic construction proposed by
Caseiro~\cite{caseiro01} and implemented by OpenGrm. In practise we 
find that the minimized LM improves the performance of the lookahead composition.
In addtion we find that pushing the labels are forward as possible can also help
the efficiency substantially.

\section{Utility Tools}
\label{sec:postprocess}
In addition to main decoding command, the OpenDcd distribution contains
many useful tools. These include the following:


\section{Kaldi Interop}
With recent importance of Kaldi as the foundation of most speech recognition
research. OpenDcd provides two levels of compatability.
When the decoder is built for standalone operation we support native reading
of Kaldi features and writing of the table format.


\section{Evaluations}
\label{sec:majhead}
In this section we demonstrate the performance of the decoder on several 
tasks.

\subsection{Wall Street Journal}

\section{Summary}
\label{sec:page}
In this paper we have described OpenDcd a toolkit. In other papers we intend to
describe the algoritms and cusomization in more depth. In future release we
plan to release one-pass neural network re-scoring.

We hope readers to will download OpenDcd from  \url{http://www.opendcd.org} and
try the easy to use scripts and command line tools with their existing Kaldi setups.
We hope that this library will become the default library and toolkit for many 
years to come.

\bibliographystyle{IEEEbib}
\bibliography{WFSTSpeech}

\end{document}


The DCD Library is a software collection for speech recognition decoding and
related functions. Based on the Finite-State Machine (FSM) Library, it provides
higher-level operations needed specifically for decoding. These include
algorithms that:

make an optimized recognition transducer (``network'') from a grammar, a
lexicon, and a context-dependency specification search a recognition transducer
to find the best matching path or a set of paths (``lattice''), given an input
utterance and an acoustic model weight lattices with just their grammatical and
acoustic components align reference and hypothesis automata using edit
distances (``scoring'') This library is not intended to provide a complete
speech recognition system: there is no acoustic model, no acoustic feature
generator (``front-end''), and no acoustic, lexical or grammatical training
provided by this package; all are considered outside the scope of this library.
It does provide a general, extensible acoustic model interface, it accepts
general context-dependency, lexical and grammatical models encoded as
finite-state transducers, and it reads pre-computed acoustic features in a
variety of formats. Our emphasis is on the generality, flexibility and
performance of the operations provided here, which are to be used as key
algorithmic components of an automatic speech recognition (ASR) system. Several
of the general operations should also find application outside of speech
recognition.  There is a program-level set of executables that is meant for use
in ``batch'' speech recognition training and testing. There is a C++
library-level set of routines that is meant for easy integration into an ASR
system. The components are quite modular and independent, so the user can
select which functionality they require.


 SEARCH MODULE ACOUSTIC MODEL
       The  search module assumes the acoustic model consists of a set of HMMs
       consecutively numbered from 1. Each  HMM h consists of  a  set  of  HMM
       states.   The  initial HMM state of h is numbered 1. There is a transi-
       tion between each HMM state i of HMM h and its successor state i+1 and,
       if  the  DSearchParams self_loop flag is set, there is also a self-loop
       transition on each HMM state.  More general  HMM  topologies  are  cur-
       rently  not  directly supported (but see Caveats). Associated with each
       HMM state is a probability density. In addition, an acoustic model  may
       support an HMM duration model or an HMM state duration model.

       The  class  DSearchModel  is the acoustic model interface for  DSearch,
       the DCD library search module (dsearch(3)). Any class correctly derived
       from  this  abstract  base  class will work with DSearch (passed in the
       DSearchParams parameter class).  The derived class's  constructor  must
       set  _hmms to point to a vector of integer vectors. Each integer vector
       represents the states in a particular HMM with the $m$-th HMM's  states
       being  stored in the top-level vector's $m-1$-th location.  The derived
       class's constructor must also set the number of distinct HMM states  in
       _nstates,  and  the flags _sdurs and _hdurs according to which, if any,
       types of duration model are provided (it would be odd but  not  prohib-
       ited  to  provide  both  types).  A data member interface rather than a
       virtual function interface is used for  these  quantities  for  maximal
       efficiency.  These protected data members can then be accessed publicly
       via several inline member functions that are provided.

       The stateCost member funtion returns the negative log  acoustic  proba-
       bility given the HMM state ID.  The hmmDurationCost (stateDurationCost)
       virtual member function returns the negative log probability given  the
       HMM ID (HMM state ID) and the model (state) duration in frames, and the
       transition type, as appropriate; the transition costs may be  duration-
       dependent  with  DSearch.   Note  that  caching  of these model values,
       important for efficient decoding, is assumed to be done by the  derived
       model class and not the search module.

       Also  note  that stateCost does not take the acoustic feature vector as
       an argument. Instead, acoustic features are passed to the model via the
       nextFrame member function of the DModel class in the next section. This
       design choice makes the search module  completely  independent  of  the
       acoustic front-end.


