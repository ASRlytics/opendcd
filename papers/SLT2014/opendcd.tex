% Template for ICIP-2012 paper; to be used with:
%          spconf.sty  - ICASSP/ICIP LaTeX style file, and
%          IEEEbib.bst - IEEE bibliography style file.
% --------------------------------------------------------------------------
\documentclass{article}
\usepackage{spconf,amsmath,graphicx,minted}

% Example definitions.
% --------------------
\def\x{{\mathbf x}}
\def\L{{\cal L}}

% Title.
% ------
\title{OpenDcd: Open Source WFST Decoding Toolkit}
%
% Single address.
% ---------------
%\name{Author(s) Name(s)}
%\address{Yandex LLC}
%
% For example:
% ------------
%\address{School\\
%	Department\\
%	Address}
%
% Two addresses (uncomment and modify for two-address case).
% ----------------------------------------------------------
%\twoauthors
%  {A. Author-one, B. Author-two\sthanks{Thanks to XYZ agency for funding.}}
%	{School A-B\\
%	Department A-B\\
%	Address A-B}
%  {C. Author-three, D. Author-four\sthanks{The fourth author performed the work
%	while at ...}}
%	{School C-D\\
%	Department C-D\\
%	Address C-D}
%
\begin{document}
%\ninept
%
\maketitle
%
\begin{abstract}
In this paper we introduce \emph{OpenDcd} a lightweight and portable Weighted Finite State
Transducer based speech decoding toolkit. The toolkit is faster and requires
substantially less memory than other open source alternatives.
\end{abstract}
%
\begin{keywords}
WFST, Decoding, Speech recognition, Open source
\end{keywords}
%
\section{Introduction}
\label{sec:intro}


Open source and reproducible results are speech recognition are an essential
part of modern science. Recently, there has become available many excellent
tools for manipulating automata, building language models, all the way up
build sophisticated speech recognitions system.  However, the state of modern 
decoding libraries is still lagging. In this paper we describe the OpenDcd release. 
One extremely important feature of the toolkit is it can be built as 
The toolkit can be built as standalone decoder with no dependencies, or as addon
with Kaldi. For the latter there is full access to all of the IO mechanims and 
acoustic models avaliable in Kaldi. Futhermore the modular nature of OpenDcd's 

\section{Introduction}
\label{sec:format}
There are several components in the system. The cascade tools, the decoder core
and the results post-processing tools.

\section{Decoder Core}
\label{sec:decodercore}

\subsection{Customization}
\label{sec:custom}

The core of the decoder is the \texttt{ArcDecoder} class that is paramterized on a transition
model, Fst  and lattice.
\begin{minted}[mathescape,
               numbersep=5pt,
               gobble=2,
               frame=lines,
               framesep=2mm]{cpp}
  template<class T, class F, class L>
  class ArcDecoder;
\end{minted}

\subsection{Lattice Generation}
The decoder supports lattice generation based on the \emph{phone-pair}
approximation and a \emph{full} lattice generation strategy.
%\url{https://code.google.com/p/shinyprofiler/}

\section{Instrumentation and Logging}
\label{sec:instrumentation}
The decoder has several built analysis mechanism, these are the call analyzer,
memory analyzer and the search space analyzer. For fined grained analysis we 
make use of the ShinyProfiler which generate very detailed call graphs.

\section{Cascade Tools}
\label{sec:cascade}

\section{Post-processing Tools}
\label{sec:postprocess}

\section{Evaluations}
\label{sec:majhead}

\subsection{Wall Street Journal}

\section{Summary}
\label{sec:page}
In this paper we have described OpenDcd a toolkit. In other papers we intend to
describe the algoritms in more depth.

\bibliographystyle{IEEEbib}
\bibliography{strings,refs}

\end{document}
